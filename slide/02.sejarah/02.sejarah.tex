\documentclass[pdflatex,compress]{beamer}

%\usetheme[dark,framenumber,totalframenumber]{ElektroITK}
\usetheme[darktitle,framenumber,totalframenumber]{ElektroITK}

\renewcommand{\figurename}{Gambar} \setbeamertemplate{caption}[numbered]

\usepackage{graphicx}
\usepackage{multicol}

\title{FORENSIKA SUARA}
\subtitle{Sejarah Audio Forensik}

\author{Tim Dosen Pengampu}

\begin{document}

\maketitle

\section{Pengantar}

\begin{frame}{Pengantar}
	\begin{itemize}
		\item Kemampuan untuk melakukan analisis audio forensik bergantung pada ketersediaan audio rekaman yang dibuat di luar studio rekaman.
		\item Alat perekam audio portabel pertama berupa pita magnetik di tahun 1950-an.
		\item Digunakan dalam rekaman saat interview, penyadapan, interogasi, dsb.
	\end{itemize}
\end{frame}

\begin{frame}{Pengantar}
	\begin{itemize}
		\item 
	\end{itemize}
\end{frame}

\section{McKeever Case}

\begin{frame}
	\frametitle{McKeever Case}
	
\end{frame}
\section{Referensi}

\begin{frame}
	\frametitle{Referensi}
	\begin{enumerate}
		\item Maher, R.C., (2018). Principles of Forensic Audio Analysis. New York: Springer.
	\end{enumerate}
\end{frame}

\end{document}
