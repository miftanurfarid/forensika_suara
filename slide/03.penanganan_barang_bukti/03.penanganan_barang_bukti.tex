\documentclass[pdflatex,compress,mathserif]{beamer}

%\usetheme[dark,framenumber,totalframenumber]{ElektroITK}
\usetheme[darktitle,framenumber,totalframenumber]{ElektroITK}

\usepackage[utf8]{inputenc}
\usepackage[T1]{fontenc}
\usepackage{lmodern}
\usepackage[bahasai]{babel}
\usepackage{amsmath}
\usepackage{amsfonts}
\usepackage{amssymb}
\usepackage{graphicx}
\usepackage{multicol}
\usepackage{lipsum}

\newcommand*{\Scale}[2][4]{\scalebox{#1}{$#2$}}%

\title{Forensika Suara}
\subtitle{Penanganan Barang Bukti}

\author{Tim Dosen Pengampu}

\begin{document}

\maketitle

\begin{frame}
	\frametitle{Sub-CPMK dan Bahan Kajian}
	\begin{itemize}
		\item \textbf{Sub-CPMK:} Mahasiswa mampu membuktikan keaslian barang bukti audio (C3)
		\item \textbf{Bahan Kajian:}
		\begin{enumerate}
			\item Identifikasi editing dalam waveform;
			\item Identifikasi editing dalam spektrum;
		\end{enumerate}
	\end{itemize}
\end{frame}

\begin{frame}
	\frametitle{Pengantar}
	\begin{itemize}
		\item Investigasi audio forensik melibatkan penanganan barang bukti.
		\item Barang bukti digital, barang bukti analog.
		\item Laboratorium forensik memiliki standar \& prosedur penanganan.
		\item Scientific Working Group on Digital Evidance (SWGDE)
	\end{itemize}
\end{frame}


\begin{frame}
	\frametitle{Basic Tools}
	\begin{enumerate}
		\item Audio playback system
		\item Waveform display program
		\item Spectrographic display program
	\end{enumerate}
\end{frame}

\begin{frame}
	\frametitle{Audio playback system}
	\begin{itemize}
		\item Frequency content
		\item Dynamic range
		\item Sampling rate dari soundcard, USB converter, dll.
		\item Speaker/ headphone dengan frequency response 50 hz - 20 kHz
		\item Headphone yang bisa mengurangi dampak background noise, room reverberation, dan yang semisal.
	\end{itemize}
\end{frame}

\begin{frame}
	\frametitle{Waveform dan Spectrographic display program}
	\begin{itemize}
		\item Audacity \href{https://www.audacityteam.org/download/}{\beamergotobutton{Link}}
	\end{itemize}
\end{frame}

\end{document}
